%% LyX 2.1.1 created this file.  For more info, see http://www.lyx.org/.
%% Do not edit unless you really know what you are doing.
\documentclass[english]{article}
\usepackage[T1]{fontenc}
\usepackage[latin9]{inputenc}
\usepackage{babel}
\begin{document}

\part{libre in the face of co-option}


\section{the logic of concern}

The default is ``closed'' -- since we consider profit to be the
driving motive, and conventionally, keeping things secret (and thus
scarce) is the key to keeping profits high for the owner.

If the driving motive is changed or modified: to usefulness, or some
moral stance, essentially ``is the product helping others?'' vs.
``is it helping me?'' then openness becomes a more reasonable justification.
Making things accessible, copy-able, less restrictive, seems the answer
to making products/information usable for the people we're interested
in helping, namely users and ``the little guys''.\\


Then, of course, co-option enters the picture. It was there before,
but at least there was some ghost of legal power to determine who
could use your stuff. In academics, of course, there was no such power
-- giving up copyright to journals meant anyone with dollars and cents
could access that information; the point is, with openness, comes
the risk that the product will be used and abused. In the case of
Linux, the U.S.' use of Linux-based system on nuclear warheads, navy
submarines, and other mechanisms of war is not exactly a \emph{good}
consequence of Linux's availability. Anthropological, ethnographic,
and other social descriptors are also used by governments and people
in power to better manipulate and control typically marginalized groups.
Point: releasing your data to run free increases the chance of the
wrong groups catching, breeding, and doing unapproved things.


\section{logic behind continuining to open}

So then what to do? One option: go back/never leave closed systems
where you exert some form of control. This, of course, is just as
vulnerable -- governments and powerful groups can bend the rules,
buy the products: if they want access, it's not hard to get, not really.
Especially with mass surveillance, technology tools, etc. So closed
source is no protection, not really, against those with power. \\


So, damned if you open, damned if you don't. For those worried about
their work being used in a way they didn't intend: no point in worrying,
it's happening no matter what. 

Thus, it seems to me that the best option is to do what you can to
help ``the little guy'', in the form of making your thing as readily
accessible and easy to use as possible. Ideally, I guess, you could
make it have no government usefulness at all -- but then it's probably
not useful to anybody else. Fighting co-option has to come from somewhere
else, whether in the form of embedding it into the software somehow
in the form of usage notes or manuals or guiding principles, but it
doesn't work to close it off. So despite the risk -- certainity --
of misuse, opening up seems like it will do more good than harm.
\end{document}
