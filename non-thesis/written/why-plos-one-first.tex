%% LyX 2.1.1 created this file.  For more info, see http://www.lyx.org/.
%% Do not edit unless you really know what you are doing.
\documentclass[twoside,english]{report}
\usepackage[T1]{fontenc}
\usepackage[latin9]{inputenc}
\usepackage{geometry}
\geometry{verbose}
\usepackage{fancyhdr}
\pagestyle{fancy}
\usepackage{babel}
\usepackage{setspace}
\onehalfspacing
\usepackage[unicode=true]
 {hyperref}
\usepackage{breakurl}

\makeatletter
\@ifundefined{date}{}{\date{}}
%%%%%%%%%%%%%%%%%%%%%%%%%%%%%% User specified LaTeX commands.
\setcounter{chapter}{1}

\makeatother

\begin{document}

\title{open science; open software; and copyright for social justice{\footnotesize{}}\\
{\large{}or why PLoSOne matters to eradicating racism}}

\maketitle
\tableofcontents{}

\pagebreak{}


\lhead{Leeper }


\section{Why PLoSOne matters to science, social justice, and techies}

Basically, intersection of
\begin{itemize}
\item open access/open science/open data free software in all its relations
to science 

\begin{itemize}
\item actual data collection (cameras, software for patch clamping) 
\item analysis of data (matlab sucks, python is better, and it's not ``just
about functionality'' in a nutshell)
\item distribution of data/papers (PDFs currently suck, and there are a
ton of better ways we could be presenting data/papers/i.e. interactive,
comments)
\item are we distributing data/papers in non-proprietary formats?) 
\item finding equity between classes/races/gender more generally, 
\end{itemize}
\item how free (access, software, data, ad. infinitum) contains the potential
for rectifying some of the vast inequalities in both science and the
world in general (both within the US and between the global North
and South)
\item simultaneouslydamage the current models are doing (proprietary software,
journal costs, ad. infinitum)
\end{itemize}



\section{framing questions}

Basically, why should ``scientists'' care about free software? Why
should the Free Software Foundation fight for open access science?
Why should people interested in helping ``third-world'' countries
and ending prejudice against the global South care about how professors
at Harvard publish their papers on quantum mechanics?u


\subsection{open science enables accountability}

Open access within science -- free, legal-to-share access to information

explicit methods

to have explicit methods, methods should be standardized to be computer
readable -- and negative results must be acceptable

There must be an acknowledgement of the biases inherent to our science,
and it seems like the only way to exorcise those biases and produce
truly better knowledge -- better, in this case, being the creation
of a model that more closely approximates and can explain data, with
the understanding that it is just a model %
\footnote{maybe? errrr%
}  -- is by opening up the processes to everyone, and making knowledge
creation a collaborative procedure. Not a given -- there can still
be scientists working by themselves, but what the work they do must
be accessible, reproducible, and explainable. Wikipedia is great;
we all use it to enhance our understanding. Reagants and analysis
software have to be accessible, for full replication of results. Data
has to be available, for re-analysis or re-interpretation or making
your own damn graphs. Your sources -- the reason you did something
-- has to be open access, so that people can go back and back and
check everything for themselves and remove our helpless dependence
on abstracts. Which isn't to say papers and abstracts won't be valuable,
but scientists need to be held accountable for the knowledge they
publish and the ways in which those publications ripple outwards.

``data collection is everything''- cherry picking!


\section[post-hoc justifications]{The post-hoc justification of research choices; final papers do
\emph{not }accurately represent the genuine process}


\subsubsection{And everyone knows it. What's up with that?}

Doesn't it do our rising students a disservice? Doesn't the academic
world spend a lot of time looking good, instead of getting down to
the hard business of things?

how does this interact with data sharing norms, with collaborative
efforts, with the fear of scoops and the necessity of getting citations?


\subsection{possible consequences - positive}

The entire process needs to be open and accessible, not intentionally
obfuscated, which will allow a number of things to happen that relate
to SJ - one, underprivileged practicitioners (underfunded universities,
DIY-scientists, fledgling programs) will be able to replicate, build
on, and use the data we have. Science will become more accountable
-- the conclusions, data, methodology, and all other aspects will
be demystified and laid bare to public scrutiny, which will allow
all sorts of people to weigh in. Feminists critiquing assumptions,
other scientists fact and method checking or modifying, data crunchers
investigating conclusions, improvement of metanalyses (allowing metaanalyses
to actually happen) Science will become more accessible -- no longer
locked in the ivory tower of privileged institutions, people -- and
i'm talking 15 year olds and 85 year olds, people from South Africa
to Iceland -- can read, access, and understand the medical establishment
that rules so much of our lives, the science behind ``oh, that's
your low cholesterol and this is why it's important.''

this also requires a fundamental shift in the intent of publishing.
While i'm not arguing for a monolithic database that catalogs research
into predetermined categories, i am arguing for a collectivist and
socialist form of science, where we give credit to information (perhaps
some sort of autotagging?), the goal will not be publication in science,
and jargon, in the derogatory sense, should be avoided and exorcised.
the intentionally vague or performative technical language should
be eradicated, if possible, and if not, there should be a service
that provides automatic services of informing the general public,
and also, goddamn, have a ``these are the abbreviations'' section.


\subsection{software}

which brings me to software -- better PDF readers that serve the needs
of a public that wants to be informed (maybe something like banshee's
wikipedia section!!), and a form of abstracts that is catalogable
and better keywords. we cannot continue to have isolated bodies of
knowledge that ARE relevant, but because they're in the ``journal
of inflammation'' people in hallucinogens never see them.

even though it IS good and relevant data.

better databases, or forms of cataloging (pubmed actually seems great
for this, although improvements like figshare or data repositories
or biotea is also really important and something i would love to see
automated and implemented on a larger scale
\begin{itemize}
\item standardized

\begin{itemize}
\item abstracts
\item data presentation
\item keywords (already around)
\item conclusions - ``we conclude this gene does this''
\end{itemize}
\end{itemize}
it does lead to the potential for codifying our way of thinking, about
genes for example...if you can just pull up everything on FoxA2 and
other gene interactions, it might prevent a different way of thinking
about--considering them in the context of pathways as opposed to their
stated functionality, maybe.


\paragraph{software shapes our thinking, software is a metaphor brought to be
something tangible and manipulatable; software is a tool in the same
way a metaphor is a tool}

\pagebreak{}


\section[Licenses]{What's Free?: Creative Commons, the GPL, and open source}

licenscing is important, and lies at the heart of all of these issues

before we go further, it is important to stifle the inevitable ``but
chrome is free!'' ``but what about freeware?'' ``what about shareware?''
``nature DOES provide a lot of its articles for free!'' ``so what
if that paper is licensced, it's free of cost! isn't that what matters?


\section{``the global South''}

propietary software locks academic and civil systems into the lockstep
of corporate computing

slightly less work in the beginning leads to a heavy load of cost
(licensing), lack of experience (no programmers to provide support,
more reliance on western support), vulnerability to the governments
that own corporations, lack of customizability/autonomny (due to the
above three).

IP law is also relevant, clearly, in realms of medicine and drugs
\begin{itemize}
\item drugs to treat HIV
\item but also necessary to have PROPER treatment => trained doctors etc.
\end{itemize}
also encourages cross culture bonding -- we're all linux users, not
bounded by territory (well, not as much) and in some ways enables
the global community that everyone claims they want?...maybe, but
also extends reach of colonizing languages -- most programming languages
are written in english - ``language of science'' -- does this make
science inherently colonizing?


\subsection{the construction of the global South as an academic consumer}
\begin{description}
\item [{link}] \href{http://www.gray-area.co.za/2012/09/04/open-access-in-africa-\%E2\%80\%93-green-and-gold-the-impact-factor-\%E2\%80\%98mainstream\%E2\%80\%99-and-\%E2\%80\%98local\%E2\%80\%99-research/}{open access in africa}
\end{description}

\section{impact factors}

\href{http://occamstypewriter.org/scurry/2012/08/13/sick-of-impact-factors/}{sick of impact factors - Stephen Curry}

\href{http://blogs.lse.ac.uk/impactofsocialsciences/2012/08/15/impact-factor-creationism-homeopathy/}{Taking the Impact factor seriously is like taking creationism seriously - Bjoern Brembs}
\end{document}
